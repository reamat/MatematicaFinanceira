\documentclass[twoside,11pt]{book}

%%%%%%%%%%%%%%%%%%%%%%%%%%%%%%%%%
%   Opcões de Linguagem
%%%%%%%%%%%%%%%%%%%%%%%%%%%%%%%%%
\usepackage[brazil]{babel}
\usepackage[utf8]{inputenc}
\usepackage[T1]{fontenc}
\usepackage{lmodern}
\usepackage{bookml/bookml}


%%%%%%%%%%%%%%%%%%%%%%%%%%%%%%%%%
% ams-latex
%%%%%%%%%%%%%%%%%%%%%%%%%%%%%%%%%
\usepackage{amsmath}
\usepackage{amssymb}
\usepackage{amsthm}
\usepackage{mathtools}


%%%%%%%%%%%%%%%%%%%%%%%%%%%%%%%%%
%  Ambientes 
%%%%%%%%%%%%%%%%%%%%%%%%%%%%%%%%%
\theoremstyle{plain}  %   bold title, italic body
\newtheorem{teo}{Teorema}[section]
\newtheorem{lem}{Lema}[section]
\newtheorem{prop}{Proposição}[section]
\newtheorem{corol}{Corolário}[section]
\newtheorem{defn}{Definição}[section]
\theoremstyle{remark}  % italic title, romman body
\theoremstyle{definition}  % italic title, romman body
\newtheorem{obs}{Observação}[section]
\newtheorem{ex}{Exemplo}[section]


%%%%%%%%%%%%%%%%%%%%%%%%%%%%%%%%%%%%%%%%%%%%%%%%%%%%%%%%%%%%%%%%%%%%%%%%%%%%
% Graphics
%%%%%%%%%%%%%%%%%%%%%%%%%%%%%%%%%%%%%%%%%%%%%%%%%%%%%%%%%%%%%%%%%%%%%%%%%%%%
\usepackage{graphics}
\usepackage{graphicx}


%%%%%%%%%%%%%%%%%%%%%%%%%%%%%%%%%%%%%%%%%%%%%%%%%%%%%%%%%%%%%%%%%%%%%%%%%%%%
% Configuração dos exercícios
% Caso livro: pacote exercise
% Caso xml: define newtheorems
%%%%%%%%%%%%%%%%%%%%%%%%%%%%%%%%%%%%%%%%%%%%%%%%%%%%%%%%%%%%%%%%%%%%%%%%%%%%
\iflatexml 
\newtheorem{ExerXML}[subsection]{E}
\newtheorem{ExeresolXML}[subsection]{ER}

\newenvironment{exer}
{\begin{ExerXML}}
{\end{ExerXML}}
\newenvironment{resp}
{\<DETAILS>\<SUMMARY>\noindent\textbf{Solução}\</SUMMARY>}
{\</DETAILS>}

\newenvironment{exeresol}
{\begin{ExeresolXML}}
{\end{ExeresolXML}}
\newenvironment{resol}
{\<DETAILS>\<SUMMARY>\noindent\textbf{Solução}\</SUMMARY>}
{\</DETAILS>}


\usepackage{chngcntr}
\
\else 
\usepackage[a4paper,headheight=15.4pt]{geometry}
\usepackage[answerdelayed,lastexercise]{exercise}
\usepackage{chngcntr}
\counterwithin{Exercise}{section}
\counterwithin{Answer}{section}
\renewcommand{\ExerciseHeaderTitle}{({\it \ExerciseTitle})}
\renewcommand{\ExerciseName}{E}
\renewcommand{\ExerciseHeader}{{\textbf{\large\ExerciseName~\ExerciseHeaderNB\ExerciseHeaderTitle\ExerciseHeaderOrigin}}}
\renewcommand{\ExerciseHeader}{\textbf{\ExerciseName\ \ExerciseHeaderNB.}\,}

% change font for answers header
\renewcommand{\AnswerHeader}{\tiny\textbf{\ExerciseName\ \ExerciseHeaderNB.}\smallskip}
% change font for answers list header
\renewcommand{\AnswerListHeader}{{\tiny\textbf{\AnswerListName\
(\ExerciseListName\ \ExerciseHeaderNB)\ ---\ }}}
%%%%%%%%%%%%%%%%%%%%%%%%%%%%%%


\newenvironment{exer}
{\begin{Exercise}}
{\end{Exercise}}

\newenvironment{resp}
{\begin{Answer}\begin{tiny}}
{\end{tiny}\end{Answer}}

\newenvironment{sol}
{\let\oldqedsymbol=\qedsymbol
  \renewcommand{\qedsymbol}{$\Diamond$}
  \begin{proof}[\bfseries\upshape Solução]}
  {\end{proof}
  \renewcommand{\qedsymbol}{\oldqedsymbol}}

%%%%%%%%%%%%%%%%%%%%%%%%%%%%%%
% Exercícios Resolvidos
%%%%%%%%%%%%%%%%%%%%%%%%%%%%%%
\newtheorem{exeresol}{ER}[section]
\newenvironment{resol}
{\let\oldqedsymbol=\qedsymbol
  \renewcommand{\qedsymbol}{$\Diamond$}
  \begin{proof}[\bfseries\upshape Solução]}
  {\end{proof}
  \renewcommand{\qedsymbol}{\oldqedsymbol}}
%%%%%%%%%%%%%%%%%%%%%%%%%%%%%%

\fi


%%%%%%%%%%%%%%%%%%%%%%%%%%%%%%%%%%%%%%%%%%%%%%%%%%%%%%%%%%%%%%%%%%%%%%%%%%%%
% Fancy Header
%%%%%%%%%%%%%%%%%%%%%%%%%%%%%%%%%%%%%%%%%%%%%%%%%%%%%%%%%%%%%%%%%%%%%%%%%%%%
\usepackage{fancyhdr}
\pagestyle{fancy}
\fancyhf{}
\fancyhead[RE]{Matemática Financeira}
\fancyhead[LO]{\rightmark}
\fancyhead[LE,RO]{\thepage}


%%%%%%%%%%%%%%%%%%%%%%%%%%%%%%%%%%%%%%%%%%%%%%%%%%%%%%%%%%%%%%%%%%%%%%%%%%%%
% Miscellaneous
%%%%%%%%%%%%%%%%%%%%%%%%%%%%%%%%%%%%%%%%%%%%%%%%%%%%%%%%%%%%%%%%%%%%%%%%%%%%
\usepackage{multicol}
\usepackage{multirow}
\usepackage[normalem]{ulem}
\renewcommand{\arraystretch}{1.5}  % space between rows in tables
\usepackage{array,booktabs}
\usepackage{xcolor}
%\usepackage{tikz}
\usepackage{microtype}
\usepackage{makeidx}
\usepackage{subfiles} % independent chapters

% license footnote
\cfoot{\tiny{Licença \href{https://creativecommons.org/licenses/by-sa/3.0/}{CC-BY-SA-3.0}. Contato: \url{reamat@ufrgs.br}}}

% no blank pages between chapters
\let\cleardoublepage\clearpage


%%%%%%%%%%%%%%%%%%%%%%%%%%%%%%%%%%%%%%%%%%%%%%%%%%%%%%%%%%%%%%%%%%%%%%%%%%%%
% MACROS E NOVOS COMANDOS
%%%%%%%%%%%%%%%%%%%%%%%%%%%%%%%%%%%%%%%%%%%%%%%%%%%%%%%%%%%%%%%%%%%%%%%%%%%%
% \DeclareTextFontCommand{\emph}{\bfseries}
\DeclareMathOperator*{\sen}{sen}
\DeclareMathOperator*{\senh}{senh}
\renewcommand{\sin}{\operatorname{\sen}}
\renewcommand{\sinh}{\operatorname{\senh}}


%%%%%%%%%%%%%%%%%%%%%%%%%%%%%%%%%%%%%%%%%%%%%%%%%%%%%%%%%%%%%%%%%%%%%%%%%%%%
% Define o ambiente abstract, que não é padrão no book
%%%%%%%%%%%%%%%%%%%%%%%%%%%%%%%%%%%%%%%%%%%%%%%%%%%%%%%%%%%%%%%%%%%%%%%%%%%%
\newcommand\summaryname{abstract}
\newenvironment{abstract}%
    {\small\begin{center}%
    \bfseries{\summaryname} \end{center}}

%%%%%%%%%%%%%%%%%%%%%%%%%%%%%%%%%%%%%%%%%%%%%%%%%%%%%%%%%%%%%%%%%%%%%%%%%%%%
% REQUIRED for correct formatting and tagging of BookML output: title and author
%%%%%%%%%%%%%%%%%%%%%%%%%%%%%%%%%%%%%%%%%%%%%%%%%%%%%%%%%%%%%%%%%%%%%%%%%%%%
\title{Matemática Financeira\\\small{Um Livro Colaborativo}}
\date{\today}
\author{}

%%%%%%%%%%%%%%%%%%%%%%%%%%%%%%%%%%%%%%%%%%%%%%%%%%%%%%%%%%%%%%%%%%%%%%%%%%%%
% RECOMMENDED for PDF quality: enable clickable links and metadata in PDFs
% the option pdfusetitle adds \author and \title to the PDF metadata
%%%%%%%%%%%%%%%%%%%%%%%%%%%%%%%%%%%%%%%%%%%%%%%%%%%%%%%%%%%%%%%%%%%%%%%%%%%%
\usepackage[pdfusetitle]{hyperref}


%%%%%%%%%%%%%%%%%%%%%%%%%%%%%%%%%%%%%%%%%%%%%%%%%%%%%%%%%%%%%%%%%%%%%%%%%%%%
% RECOMMENDED for PDF quality: higher quality equivalent of Computer Modern
%%%%%%%%%%%%%%%%%%%%%%%%%%%%%%%%%%%%%%%%%%%%%%%%%%%%%%%%%%%%%%%%%%%%%%%%%%%%
\usepackage{lmodern}


%%%%%%%%%%%%%%%%%%%%%%%%%%%%%%%%%%%%%%%%%%%%%%%%%%%%%%%%%%%%%%%%%%%%%%%%%%%%
% REQUIRED for TikZ: use bmlImageEnvironment for TikZ and other images, since
%                    LaTeXML is slow and often produces garbled TikZ output
%                    (this requires \usepackage{bookml/bookml})
% tell BookML to compile tikzpicture and tikzcd environments into images via LaTeX

\bmlImageEnvironment{tikzpicture,tikzcd}
% hide all TikZ-related commands from LaTeXML, but leave them when generating the PDF
\iflatexml            % are we compiling through LaTeXML?
\else
\usepackage{tikz}     % if not, load TikZ as usual
\usetikzlibrary{cd}
\usepackage[all]{xy}
% other TikZ-related command MUST be here, before \fi
\fi
%%%%%%%%%%%%%%%%%%%%%%%%%%%%%%%%%%%%%%%%%%%%%%%%%%%%%%%%%%%%%%%%%%%%%%%%%%%%


%%%%%%%%%%%%%%%%%%%%%%%%%%%%%%%%%%%%%%%%%%%%%%%%%%%%%%%%%%%%%%%%%%%%%%%%%%%%
%%% OPTIONAL for accessibility: additional versions of the document (e.g. large print)
% the additional formats will be recompiled automatically whenever there are changes
% you may use "\jobname" instead of "template" to pick up the name from this very file
% % \bmlAltFormat{template.pdf}{PDF (serif)} % always included; here \bmlAltFormat simply changes its label
% % \bmlAltFormat{template-sans.pdf}{PDF (sans serif)}
% % \bmlAltFormat{template-sans-large.pdf}{PDF (sans, large)}

% very cheap way of generating a 'large print' PDF (~19% bigger)
\ifcsname bmlCrop\endcsname % is \bmlCrop defined?
\usepackage{crop}           % if so, load the package crop
\fi
%%% What is \bmlCrop?
% template-sans-large.tex uses \input{template.tex} to read this file.
% Before that, it defines \bmlCrop, so that the above code imports crop.
% It also uses \PassOptionsToPackage to tweak the options of crop.
% The result has identical pagination as the normal PDF so you do not need
% to review both versions.
%
% However, the magnification is very limited. If you need larger magnification,
% you may have to use a larger font, in which case the large print PDF will
% will look different and must be reviewed regularly.

%%%%%%%%%%%%%%%%%%%%%%%%%%%%%%%%%%%%%%%%%%%%%%%%%%%%%%%%%%%%%%%%%%%%%%%%%%%%



\begin{document}
\iflatexml
\begin{abstract}
  % RECOMMENDED for SCORM: the abstract becomes the description of the SCORM package
  % Copiei do prefácio
  Este livro busca abordar os tópicos de um curso moderno de Matemática Financeira.
\end{abstract}
\fi


\maketitle

\mainmatter
\include{organizadores}
\include{licenca}
\include{nota_organizadores}
\include{prefacio}


% %%% OPTIONAL: footer for HTML output, for instance a copyright notice
% % note that the footer does not appear in the PDF
% % the PDF footer can be done in any of the usual ways (e.g. fancyhdr)
\iflatexml

\begin{lxFooter}
  Licença CC BY-SA 3.0 http://creativecommons.org/licenses/by-sa/3.0/
\end{lxFooter}
\fi
% Capítulos

\mainmatter
 
%Este trabalho está licenciado sob a Licença Creative Commons Atribuição-CompartilhaIgual 3.0 Não Adaptada. Para ver uma cópia desta licença, visite http://creativecommons.org/licenses/by-sa/3.0/ ou envie uma carta para Creative Commons, PO Box 1866, Mountain View, CA 94042, USA.
%!TEX root = ../livro.tex

\newpage
\setcounter{section}{1}
\section{Juros Simples}
%\autores

No tópico anterior, estudamos exemplos de operações financeiras em que a capitalização dos juros acontecia uma única vez ao final da operação. Mas na maioria das operações de investimentos ou empréstimos, a capitalização acontece em períodos diferentes da duração da operação. Por exemplo, um capital investido na Caderneta de Poupança pode não receber juros (se retirado antes de um mês) ou receber juros mensais. Também, um empréstimo tomado por um prazo pode ser quitado parcial ou integralmente antes da data original, reduzindo assim o valor dos juros da operação, e fazendo-se necessário que se saiba como a dívida evolui diariamente/mensalmente. Em ambas as situações, precisamos saber como fracionar o cálculo do juros para outro período (meses no caso da poupança e dias para o empréstimo) e analisar como se dá a evolução do capital investido com a agregação de juros por período. 

A seguir, veremos o caso mais simples de capitalização periódica de juros.


\subsection{Capitalização a juros simples}

No regime de capitalização simples, os juros são calculados somente sobre o valor aplicado, ou seja, a taxa de juros incide somente sobre o principal. Assim, o valor dos juros por cada período é constante, pois só depende do principal e da taxa. Portanto, se a taxa de juros em cada período é $i$, temos que os juros em cada período será dado por $i\cdot P$.

Considere um empréstimo de R\$ 1.000,00 por 3 meses sujeito à taxa de juros simples de 20\% a.m. Ao final do primeiro mês, os juros devidos serão 20\% do capital emprestado, ou seja, $J_1 = 0,2 \cdot 1000 = 200$, e a dívida neste instante é de $S_1= P+J_1 = 1200$. Transcorrido mais um mês, os novos juros devidos serão 20\% do capital emprestado, ou seja, $J_2 = 200$, e a dívida neste momento é de $S_2=P+J_1+J_2=1400$. Ao final do terceiro mês, os juros novamente serão sobre o capital originalmente emprestado, e assim $J_3 = 200$. Assim, os juros gerados pelo empréstimo foram de $J=J_1+J_2+J_3=3\cdot 200=600$, e portanto, a dívida a ser paga é de R\$ 1.600. 



\clearpage
\appendix
%\include{./apendice/apendice}

% Respostas dos exercícios no final do livro
\iflatexml
\else
\input{respostas} 
\fi


% Referências bibliográficas
\nocite{*}
\bibliographystyle{plain}
\bibliography{main}
\addcontentsline{toc}{chapter}{Referências Bibliográficas}
% \fancyhead[RE]{Transformada de Laplace}
% \fancyhead[LO]{REFERÊNCIAS BIBLIOGRÁFICAS}
% \fancyhead[LE,RO]{\thepage}



\backmatter
%Este trabalho está licenciado sob a Licença Creative Commons Atribuição-CompartilhaIgual 3.0 Não Adaptada. Para ver uma cópia desta licença, visite https://creativecommons.org/licenses/by-sa/3.0/ ou envie uma carta para Creative Commons, PO Box 1866, Mountain View, CA 94042, USA.
% !TEX root = ../livro.tex
\chapter*{Colaboradores}\label{colaboradores}
\iflatexml\else\addcontentsline{toc}{chapter}{Colaboradores}\fi

Este material é fruto da escrita colaborativa. Veja a lista de colaboradores em: 
ACRESCENTE AQUI O LINK PARA A LISTA DE COLABORADORES 

\begin{center} 
  \url{https://github.com/reamat/MatematicaFinanceira/graphs/contributors}
\end{center}

Para saber mais como participar, visite o site oficial do projeto:
\begin{center}
  \url{https://www.ufrgs.br/reamat/MatematicaFinanceira}
\end{center}
ou comece agora mesmo visitando nosso repositório GitHub:
\begin{center}
  \url{https://github.com/reamat/MatematicaFinanceira}
\end{center}
\printindex

\end{document}
